\documentclass{article}
\usepackage{amsmath}
\usepackage{amssymb}
\usepackage{geometry}
\usepackage{graphicx}
\geometry{margin=1in}

\title{Collision-Aware Velocity Shaping (2D Mockup)}
\author{roballgame}
\date{}

\begin{document}
\maketitle

\section*{State and Geometry}
We model a 2D point mass (dot) of radius $r$ at position $p \in \mathbb{R}^2$.
The user provides a joystick velocity (or acceleration) command $v_{cmd} \in \mathbb{R}^2$.
Obstacle $i$ yields a signed distance $d_i$ and outward unit normal $n_i$ at the closest point.
We use the nearest obstacle $(d, n)$ unless stated otherwise.

\paragraph{Circle obstacle.} For a circle centered at $c$ with radius $R$:
\[
 d = \lVert p - c \rVert - (R + r), \qquad n = \frac{p - c}{\lVert p - c \rVert}.
\]

\paragraph{Axis-aligned walls.} For boundaries: left, right, top, bottom.
\[
 d_{left} = p_x - r, \quad n_{left} = (1,0) 
\]
\[
 d_{right} = (W - p_x) - r, \quad n_{right} = (-1,0)
\]
\[
 d_{top} = p_y - r, \quad n_{top} = (0,1)
\]
\[
 d_{bottom} = (H - p_y) - r, \quad n_{bottom} = (0,-1)
\]

\paragraph{Axis-aligned rectangles.} For a rectangle with top-left $(x,y)$ and size $(w,h)$,
expand by $r$ and compute the distance to the nearest point on the expanded rectangle.
Let $L=x-r$, $R=x+w+r$, $T=y-r$, $B=y+h+r$ and clamp $q=(p_x,p_y)$ into
$\tilde{q}=(\mathrm{clamp}(p_x,L,R), \mathrm{clamp}(p_y,T,B))$.
Then $d=\lVert q-\tilde{q}\rVert$ and $n=(q-\tilde{q})/\lVert q-\tilde{q}\rVert$.
If $q$ is inside the expanded rectangle, $d$ is negative and $n$ points to the nearest side.

\paragraph{Line segments.} For a segment from $a$ to $b$, project $p$ onto the segment:
\[
t = \mathrm{clamp}\left( \frac{(p-a)\cdot(b-a)}{\lVert b-a\rVert^2}, 0, 1 \right), \quad
c = a + t (b-a)
\]
\[
d = \lVert p - c \rVert - r, \quad n = \frac{p - c}{\lVert p - c \rVert}.
\]

\section*{Model 1: Speed Scaling}
Define a scale factor $s(d)$ that shrinks as the distance to obstacles decreases (Figure~\ref{fig:speed-scaling}).
\[
 s(d) = \mathrm{clamp}\left( \frac{d - d_{stop}}{d_{slow} - d_{stop}}, 0, 1 \right)
\]
\[
 v_{safe} = s(d)\, v_{cmd}.
\]
\begin{figure}[ht]
\centering
\includegraphics[width=0.9\linewidth]{figures/speed_scaling.png}
\caption{Speed scaling factor used in Models 1 and 4.}
\label{fig:speed-scaling}
\end{figure}

\section*{Model 2: Repulsive Field}
Compute a repulsive velocity from nearby obstacles and add it to the command (Figure~\ref{fig:repulsion}).
For each obstacle with $d_i < d_0$:
\[
 v_{rep,i} = k \left( \frac{1}{d_i} - \frac{1}{d_0} \right) \frac{1}{d_i^2} \, n_i,
\]
and $v_{rep,i}=0$ for $d_i \ge d_0$. Optionally clamp $\lVert v_{rep,i} \rVert$.
\[
 v_{safe} = v_{cmd} + \sum_i v_{rep,i}.
\]
\begin{figure}[ht]
\centering
\includegraphics[width=0.9\linewidth]{figures/repulsion.png}
\caption{Repulsive field magnitude for Model 2 (example parameters).}
\label{fig:repulsion}
\end{figure}

\section*{Model 3: Normal Projection}
Project out motion that pushes into the nearest obstacle surface (Figure~\ref{fig:normal-projection}).
\[
 v_n = (v_{cmd} \cdot n)\, n
\]
If $v_{cmd}$ points into the obstacle, remove that component:
\[
 v_{safe} = v_{cmd} - \max(0, -v_{cmd}\cdot n)\, (-n)
\]
Equivalently, if $v_{cmd}\cdot n < 0$, then
\[
 v_{safe} = v_{cmd} + (v_{cmd}\cdot n)\, n.
\]
\begin{figure}[ht]
\centering
\includegraphics[width=0.9\linewidth]{figures/normal_projection.png}
\caption{Normal projection (Model 3): motion into the surface is removed.}
\label{fig:normal-projection}
\end{figure}

\section*{Model 4: Damped Barrier}
Decompose command into normal and tangential components, then damp only the normal component (Figure~\ref{fig:damped-barrier}).
\[
 v_n = (v_{cmd}\cdot n)\, n, \quad v_t = v_{cmd} - v_n
\]
\[
 v_{safe} = s(d)\, v_n + v_t
\]
with the same $s(d)$ as Model 1.
\begin{figure}[ht]
\centering
\includegraphics[width=0.9\linewidth]{figures/damped_barrier.png}
\caption{Normal component scaling in the damped barrier (Model 4).}
\label{fig:damped-barrier}
\end{figure}

\section*{Logging}
In the mockup, we log $(t, d, \lVert v \rVert, \lVert v_{cmd} \rVert)$ each frame to
enable comparison of how each model shapes velocity vs. distance.

\end{document}
